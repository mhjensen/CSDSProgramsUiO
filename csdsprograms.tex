\documentclass[oneside,final,10pt]{article}

\listfiles               %  print all files needed to compile this document

\usepackage{relsize,makeidx,color,setspace,amsmath,amsfonts,amssymb}
\usepackage[table]{xcolor}
\usepackage{bm,ltablex,microtype}
\usepackage[pdftex]{graphicx}

\usepackage[T1]{fontenc}
%\usepackage[latin1]{inputenc}
\usepackage{ucs}
\usepackage[utf8x]{inputenc}

\usepackage{lmodern}         % Latin Modern fonts derived from Computer Modern

% Hyperlinks in PDF:
\definecolor{linkcolor}{rgb}{0,0,0.4}
\usepackage{hyperref}
\hypersetup{
    breaklinks=true,
    colorlinks=true,
    linkcolor=linkcolor,
    urlcolor=linkcolor,
    citecolor=black,
    filecolor=black,
    %filecolor=blue,
    pdfmenubar=true,
    pdftoolbar=true,
    bookmarksdepth=3   % Uncomment (and tweak) for PDF bookmarks with more levels than the TOC
    }

\setcounter{tocdepth}{2}  % levels in table of contents

\setcounter{topnumber}{2}
\setcounter{bottomnumber}{2}
\setcounter{totalnumber}{4}
\renewcommand{\topfraction}{0.95}
\renewcommand{\bottomfraction}{0.95}
\renewcommand{\textfraction}{0}
\renewcommand{\floatpagefraction}{0.75}
\clubpenalty = 10000
\widowpenalty = 10000

\raggedbottom
\makeindex
\usepackage[totoc]{idxlayout}   % for index in the toc
\usepackage[nottoc]{tocbibind}  % for references/bibliography in the toc

\begin{document}

\thispagestyle{empty}

\begin{center}
{\LARGE\bf
\begin{spacing}{1.25}
Computing Across the Disciplines: Study Programs in Computational Science and Data Science
\end{spacing}
}
\end{center}

\begin{center}
{\bf A comprehensive path from undergraduate studies to graduate studies (Master and PhD levels) in Computational Science and Data Science at the University of Oslo }\\ [0mm]
\end{center}


\vspace{1cm}


\section*{Executive Summary}

We propose to establish, with startup Fall semester 2023, a new bachelor/undergraduate program in Computational Science and Data Science at the University of Oslo. 
The new undergraduate program is meant to be a multidisciplinary program with  a strong methodological base of courses in Mathematics, Statistics, Computational Science and Data Science, totalling 90-100 ECTS. The remaining credits will allow students to specialize in different disciplines, like Bioscience, Chemistry, Geoscience, Informatics, Physics and other physical sciences. The program's multidisciplinary nature entails thus a close collaboration between departments at the Faculty of Mathematics and Natural Science.  

We propose also that the two existing Master of Science programs, Computational Science and Data Science, merge into one Master of Science program with the same name as the undergraduate program. The new Master of Science program will include all the study directions from the previous two Master of Science programs. Planned startup for the new program is also Fall 2023. 

Combined with the newly started {\bf CompSci} PhD program, our proposal is meant to enhance the visibility of Computational Science and Data Science at the University of Oslo, from undergraduate to graduate studies. It will also align well with many of the new research and educational initiatives on Computational Science and Data Science at the university of Oslo, in particular the newly established {\bf dScience} center.



\section*{Introduction and Background}

Computational Science and Data Science play a central role in scientific investigations and are central to innovation in most domains of our lives. These fields underpin the majority of today's technological, economic and societal feats. We have entered an era in which huge amounts of data offer enormous opportunities, but only to those who are able to harness them. The 3rd Industrial Revolution will alter significantly the demands on the workforce. To adapt a highly-qualified workforce to upcoming challenges requires strong fundamental bases in STEM fields. Computational Science and Data Science can provide such bases at all stages. Most of our students at both the undergraduate and the graduate level are unprepared to use computational modelling, data science, and high performance computing – skills valued by a very broad range of employers. 
The above developments, needs and future challenges, as well as the exciting developments which are now taking place within quantum computing, quantum information and data driven discoveries (data analysis and machine learning) will play an essential role in shaping future technological developments. Most of these developments require true cross-disciplinary approaches.  In order to meet many of these challenges we propose here that the Faculty of Mathematics and Natural Science at the University of Oslo, establishes, with start Fall 2023, a new undergraduate program in Computational Science and Data Science.

Furthermore, we propose also that the two Master of Science programs, Computational Science and Data Science merge into one programs. This will enhance the visibility of the two fields and most likely attract many new students. Combined with the newly established {\bf CompSci} PhD program, this has the potential to offer new educational research paths in computational science and data science. 


\section*{Educational programs in Computational Science and Data Science}


\subsection*{Bachelor of Science in Computational Science and Data Science}

The new undergraduate program in Computational Science and Data Science will offer a basic training in central elements in Mathematics, Computational Science, Data Science, Statistics and Informatics (programming and system development). 
The core part of the program is made up of central courses (see description in the appendix) in the above fields. Most of these courses are already available, while some courses will have to be developed in close collaboration with the involved departments at the Faculty for Mathematics and Natural Science.
The core block of courses is set to 90-100 ECTS and will be compulsory for all students entering the program. These courses will also be the main requirements for admission to
 all study directions in the new Computational Science and Data Science Master of Science program.


The remaining 80 ECTS are elective and students can select from a large range of disciplines such as Mathematics, Statistics, Physics, Chemistry, Geoscience, Bioscience, Astrophysics and Informatics. 
%Recommended blocks of courses, qualifying for additional master programmes will be provided.
With the core block of courses combined with recommended combinations of additional courses,  students should be able to apply for many Master of Science programs at the Faculty for Mathematics and Natural Science. 
%This program will also allow students to be qualified for all study directions in the new Computational Science and Data Science Master of Science program.

The compulsory courses are listed in the appendix.

\subsection*{Master of Science in Computational Science and Data Science}

Ting å diskutere i dokumentet.
Legg til om anvendelses del i CS og metode i DS.

Presently the Faculty of Mathematics and Natural Science at the University of Oslo offers two Master of Science programs. One in Computational Science with ten study directions and one in Data Science with five specialisations. In order to enhance the visibility of Computational Science and Data Science we propose to merge these two programs into one program and maintaining the study directions and specialisations. Within the new program we propose two methodological directions towards computational science and data science, respectively in addition to a more applied direction, combining both computational science and data science tools towards other fields. Compared to the existing master program in Computational Science, the new program will have a stronger focus on methodology while at the same time keeping the existing 10 study directions within the applied direction. Compared to the existing master program in Data Science, the methodological focus is kept but with the additional possibility of more applied directions.

Legg til om dScience og PhD.  


%\subsection*{PhD in Computational Science and Data Science}
%This needs to be discussed.









\section*{Appendix:  Outline of degree programs and courses}

\subsection*{Themes for a CS+DS Bachelor of Science program}

Here we have a list of themes which could enter the common part of the CS+DS bachelor program.  
Semesters are not indicated. Courses could also be modularized in 5 ECTS units. Courses in red are compulsory.



\paragraph{Mathematics}
Here we can think of the topics and themes covered by the central  Math courses we have presently.
30 ECTS in total
\begin{enumerate}
\color{red}
    \item \textcolor{red}{MAT1100}, 1st year
    \item \textcolor{red}{MAT1110}, 1st year
    \item \textcolor{red}{MAT1120}, 2nd year
\end{enumerate}

\paragraph{Programming and system development}
Here we can think of the topics and themes covered by the existing central programming courses.
20 ECTS plus 10 elective ECTS (also smaller 5 ECTS modules)
\begin{enumerate}
    \item \textcolor{red}{IN1900} 1st year
    \item \textcolor{red}{IN1910}
    \item IN3200, High-performance Computing, elective
\end{enumerate}

GKS: Like mathematics, informatics is also a domain-agnostic "methods" field (with a few exceptions, like courses on computers and operating systems per se). However, these methods topics cover a very broad range, and are distributed across a large number of courses. While a student of computational/data science might expect to gain a basic competence in such topics, at least to be well prepared to collaborate with pure computer scientists, it may not make sense to go to the same depth on these topics as IFI students. Thus, we also discussed whether it might be useful to define a single new course that at least gives CS/DS-students a feeling of these topics. The following is a very rough suggestion, which would need to be both pruned and improved, but hopefuly useful as a basis for discussion:
\paragraph{Course title, maybe: "Systems engineering for data science"}
\paragraph{Candidate theoretical topics (method):}
\begin{enumerate}
\item Software development methodologies: mainly agile!?
\item Software architecture/design: How do you plan and communicate the design of a data science codebase before and during implementation (with colleagues and customers)
\item Software maintainability: how to develop code that stand the test of time - allowing the code to be updated to future requirements and extended with new features of for new applications (also include code reuse)
\item Software testing: how do you ensure confidence that the code works as intended and provides valid results (unit testing, effective debugging)
\item Databases: brief introduction to different types of database systems of relevance to data science - relational databases, object databases, flat file databases. And brief intro to database normal forms.
\item Basic parallelization approaches for data science: brief intro to multithreading, multiprocessing and MPI. Also approaches like map-reduce.
\item Brief intro to data structures and algorithms for data scientists: the algorithmic approach to solving problems, pseudo-code, asymptotic time complexity, ...
\item Brief intro to user experience design
\item Brief intro to information security and data privacy
\item Brief into to semantic technologies
\item Brief into to domain-specific languages - what it is, why it is useful and how one define one's own embedded or external tailored DSLs.
\end{enumerate}

\paragraph{Candidate practical topics}
\begin{enumerate}
\item Analysis reproducibility and FAIR data
\item Data science frameworks for scaling analyses, like map-reduce, Hadoop, Spark, Ray, ...
\item Difference between programming languages
\item Main libraries for data science - scikitlearn, scipy, pytorch, tensorflow, keras, R, ...
\item Frameworks for data and code sharing - Jupyter, Github (including git and version control) ..
\item Brief intro to use of computer and HPC systems - memory and storage models for single machines versus clusters, queueing systems, virtual machines and cloud systems, Docker containers, virtual environments, package managers
\item Becoming a productive programmer: techniques for efficient programming in high-level languages 
\item Programming for efficiency: brief into to computational efficiency of different programming languages, efficiency of different programming constructs/techniques, memory- and cache-efficiency..
\end{enumerate}

\paragraph{Computational Science}
20 ECTS + 10 ECTS elective (also smaller 5 ECTS modules)
\begin{enumerate}
    \item \textcolor{red}{Numerical Methods I} Could be revised version of MAT-INF1100, first year
    \item \textcolor{red}{Numerical Methods II}, second year, could be MAT3110
    \item Numerical Methods III, elective,  third year, FYS3150, Computational Physics, third year
\end{enumerate}

\paragraph{Probability and Statistics}
20 ECTS + 10 ECTS elective (also smaller 5 ECTS modules). Here we can think of courses tailored to specific disciplines as well. 

 In order to understand methods within data science (or perhaps here restricted to data analysis), uncertainty, random variables and probability are important concepts that needs to be learning. For most students these are new concepts that they use time to learn properly. 
\begin{enumerate}
    \item \textcolor{red}{Introduction to probability and statistics}, first year, preferable a revised version of STK1100.
    
    An introductory course (10 ECTS) should introduce probability theory (including conditional probabilities and Bayes theorem, some common distributions, expectations, variances, covariances), statisitcal inference (the likelihood function, parametric modelling, ML estimation, bias in estimation, confidence intervals Bootrapping, some simple models). In order to make the course attractive for several bachelor programs, the course should include good examples/case studies from substance disciplines.
    \item \textcolor{red}{Statistical learning and machine learning}, 2nd year, preferable STK2100
    
    A second course (10 ECTS) should focus on statistical/machine learning methods, starting with linear models (going a bit in depth on this) and then demonstrating \emph{some} advanced methods (e.g nearest neighbor, neural network). Possibly also examples of unsupervised learning/data mining.

    \item Data Analysis, AI and Statistics III, third year
    
    Possible courses here are STK3100 - Generalized linear models,  
IN3050 – Introduction to Artificial Intelligence and Machine Learning, FYS-STK3155 – Applied Data Analysis and Machine Learning 
\end{enumerate}




MI (section for statistics and Data Science) is currently looking at modifications of STK1100 and STK2100 following the ideas above. A main change from the current set of courses is that STK2100 can be taken with only STK1100 as background (today it also is built on STK1110).

\paragraph{Discipline oriented courses}

These can be specific courses that lead to the requirements of a bachelor degree in life science, physics, chemistry etc. Most bachelor programs have a requirement of 60-90 ECTS within the specific fields. As an example, the Physics and Astronomy bachelor requires 60 ECTS in mathematics and programming and 70 ECTS in physics courses, pluss 10 ECTS in advanced topics and 40 elective ECTS.  
Chemistry and Biochemistry requires 100 ECTS. 





\end{document}


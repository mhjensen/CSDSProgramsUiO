\documentclass[oneside,final,10pt]{article}

\listfiles               %  print all files needed to compile this document

\usepackage{relsize,makeidx,color,setspace,amsmath,amsfonts,amssymb}
\usepackage[table]{xcolor}
\usepackage{bm,ltablex,microtype}
\usepackage[pdftex]{graphicx}

\usepackage[T1]{fontenc}
%\usepackage[latin1]{inputenc}
\usepackage{ucs}
\usepackage[utf8x]{inputenc}

\usepackage{lmodern}         % Latin Modern fonts derived from Computer Modern

% Hyperlinks in PDF:
\definecolor{linkcolor}{rgb}{0,0,0.4}
\usepackage{hyperref}
\hypersetup{
    breaklinks=true,
    colorlinks=true,
    linkcolor=linkcolor,
    urlcolor=linkcolor,
    citecolor=black,
    filecolor=black,
    %filecolor=blue,
    pdfmenubar=true,
    pdftoolbar=true,
    bookmarksdepth=3   % Uncomment (and tweak) for PDF bookmarks with more levels than the TOC
    }

\setcounter{tocdepth}{2}  % levels in table of contents

\setcounter{topnumber}{2}
\setcounter{bottomnumber}{2}
\setcounter{totalnumber}{4}
\renewcommand{\topfraction}{0.95}
\renewcommand{\bottomfraction}{0.95}
\renewcommand{\textfraction}{0}
\renewcommand{\floatpagefraction}{0.75}
\clubpenalty = 10000
\widowpenalty = 10000

\raggedbottom
\makeindex
\usepackage[totoc]{idxlayout}   % for index in the toc
\usepackage[nottoc]{tocbibind}  % for references/bibliography in the toc

\begin{document}

\thispagestyle{empty}

\begin{center}
{\LARGE\bf
\begin{spacing}{1.25}
Computing Across the Disciplines: Study Programs in Computational Science and Data Science
\end{spacing}
}
\end{center}

\begin{center}
{\bf A comprehensive path from undergraduate studies to graduate studies (Master and PhD levels) in Computational Science and Data Science at the University of Oslo }\\ [0mm]
\end{center}


\vspace{1cm}


\section*{Executive Summary}
We propose to establish, with startup Fall semester 2023, a new bachelor/undergraduate program in Computational Science and Data Science at the University of Oslo. This program will be hosted by the Department of Mathematics in close collaboration with other departments at the Faculty of Mathematics and Natural Science.

We propose also that the two existing Master of Science programs, Computational Science and Data Science, merge into one Master of Science program with the same name as the undergraduate program. The new Master of Science program will include all the study directions from the previous two Master of Science programs. Planned startup for the new program is also Fall 2023. This program will also be hosted by the Department of Mathematics.

The new undergraduate program is meant to be a multidiscipalinary with  a strong methodological base of courses in Mathematics, Statistics, Computational Science and Data Science, totalling 90-100 ECTS. The remaining credits allow students to specialize in different disciplines, like Bioscience, Chemistry, Geoscience, Informatics, Physics and other physical sciences. 

\section*{Introduction and Background}
Computational Science and Data Science play a central role in scientific investigations and are central to innovation in most domains of our lives. These fields underpin the majority of today's technological, economic and societal feats. We have entered an era in which huge amounts of data offer enormous opportunities, but only to those who are able to harness them. The 3rd Industrial Revolution will alter significantly the demands on the workforce. To adapt a highly-qualified workforce to coming challenges requires strong fundamental bases in STEM fields. Computational Science and Data Science can provide such bases at all stages. Most of our students at both the undergraduate and the graduate level are unprepared to use computational modeling, data science, and high performance computing – skills valued by a very broad range of employers. 
The above developments, needs and future challenges, as well as the exciting developments which are now taking place within quantum computing, quantum information and data driven discoveries (data analysis and machine learning) will play an essential role in shaping future technological developments. Most of these developments require true cross-disciplinary approaches.  In order to meet many of these challenges we propose here that the Faculty of Mathematics and Natural Science at the University of Oslo, establishes, with start Fall 2023, a new undergraduate program in Computational Science and Data Science


\section*{Educational programs in Computational Science and Data Science}


\begin{enumerate}
\item An undergraduate program in Computational Science and Data Science.

\item We have already (from fall 2018) two new Master of Science programs in Computational Science and Data Science dailored to STEM fields. Should we merge these two programs into one and have one coordinating unit? 

\item Develop a PhD program in Computational Science and Data Science.

\subsection*{Bachelor of Science in Computational Science and Data Science}

\subsection*{Master of Science in Computational Science and Data Science}

\subsection*{PhD in Computational Science and Data Science}


\end{enumerate}










\section*{Appendix:  Outline of degree programs and courses}

\subsection*{Themes for a CS+DS Bachelor of Science program}

Here we have a list of themes which could enter the common part of the CS+DS bachelor program.  
Semesters are not indicated. Courses could also be modularized in 5 ECTS units. Courses in red are compulsory.



\paragraph{Mathematics}
Here we can think of the topics and themes covered by the central  Math courses we have presently.
30 ECTS in total
\begin{enumerate}
\color{red}
    \item MAT1100
    \item MAT1110
    \item MAT1120
\end{enumerate}

\paragraph{Programming}
Here we can think of the topics and themes covered by the existing central programing courses.
20 ECTS plus 10 elective ECTS (also smaller 5 ECTS modules)
\begin{enumerate}
    \item \textcolor{red}{IN1900} 1st year
    \item \textcolor{red}{IN1910}
    \item IN3200, HPC, elective
\end{enumerate}


\paragraph{Computational Science}
20 ECTS + 10 ECTS elective (also smaller 5 ECTS modules)
\begin{enumerate}
    \item \textcolor{red}{Numerical Methods I} Could be revised version of MAT-INF1100, first year
    \item \textcolor{red}{Numerical Methods II}, second year
    \item Numerical Methods III, elective, MAT3110, third year
\end{enumerate}

\paragraph{Data Science}
20 ECTS + 10 ECTS elective (also smaller 5 ECTS modules). Here we can think of courses tailored to specific disciplines as well. 
\begin{enumerate}
    \item \textcolor{red}{Data Analysis, AI and Statistics I}, first year
    \item \textcolor{red}{Data Analysis, AI and Statistics II}, 2nd year
    \item Data Analysis, AI and Statistics III, third year
\end{enumerate}

Geir: In order to understand methods within data science (or perhaps here restricted to data analysis), uncertainty, random variables and probability are important concepts that needs to be learning. For most students these are new concepts that they use time to learn properly. 

An introductory course (10 ECTS) should introduce probability theory (including conditional probabilities and Bayes theorem, some common distributions, expectations, variances, covariances), statisitcal inference (the likelihood function, parametric modelling, ML estimation, bias in estimation, confidence intervals Bootrapping, some simple models). The course should \emph{not} be module based, but include good examples/case studies from sustance disciplines.

A second course (10 ECTS) should focus on statistical/machine learning methods, starting with linear models (going a bit in depth on this) and then demonstrating \emph{some} advanced methods (e.g nearest neighbor, neural network). Possible also examples of unsupervised learning.

MI (section for statistics and Data Science) is currently looking at modifications of STK1100 and STK2100 following these ideas. A main change from the current set of courses is that STK2100 can be taken with only STK1100 as background (today it also is built on STK1110).

\paragraph{Discipline oriented courses}

These can be specific courses that lead to the requirements of a bachelor degree in life science, physics, chemistry etc. Most bachelor programs have a requirement of 60-90 ECTS. As an example, the Physics and Astronomy bachelor requires 60 ECTS in mathematics and programming and 70 ECTS in physics courses, pluss 10 ECTS in advanced topics and 40 elective ECTS.  
Chemistry and Biochemistry requires 100 ECTS. 

\paragraph{Degree programs.}

The MNFak  offers from fall 2018 two new programs at the Master of Science level in Computational Science and Data Science. These programs are
\begin{enumerate}
\item \href{{http://www.uio.no/english/studies/programmes/computational-science-master/index.html}}{Computational Science}, start fall 2018

\item \href{{http://www.uio.no/english/studies/programmes/datascience-master/index.html}}{Data Science}, start fall 2018

\item We may consider merging these two programs and have one coordinating unit.
\item Develop an all university PhD program in Computational Science and Data Science by fall 202X?


\

\item Develop a bachelor program in Computational Science and Data Science by 2023.
\end{enumerate}

\noindent

\paragraph{Courses.}
There are several existing and planned courses which could be offered by the new department.
These are:
\begin{enumerate}

\item New courses on advanced data analysis and machine learning including for example:

\noindent
Many of these courses, if properly modularized, can be offered as intensive training courses and programs. In particular, such courses will be attractive for both the private and public sectors. The following courses could be offered
\begin{enumerate}
\item Introductory Scientific Python

\item Advanced Scientific Python

\item Data Science and visualization

\item Applied numerical mathematics

\item Computational finance

\item Big data graph analysis

\item Supervised machine learning with scikit-learn and TensorFlow

\item Unsupervised machine learning with scikit-learn

\item Data-driven entrepreneurship

\item Courses tailored to the needs of specific companies

\item and more specialized modules
\end{enumerate}


\end{document}


\documentclass[oneside,final,10pt]{article}

\listfiles               %  print all files needed to compile this document

\usepackage{relsize,makeidx,color,setspace,amsmath,amsfonts,amssymb}
\usepackage[table]{xcolor}
\usepackage{bm,ltablex,microtype}
\usepackage[pdftex]{graphicx}

\usepackage[T1]{fontenc}
%\usepackage[latin1]{inputenc}
\usepackage{ucs}
\usepackage[utf8x]{inputenc}

\usepackage{lmodern}         % Latin Modern fonts derived from Computer Modern

% Hyperlinks in PDF:
\definecolor{linkcolor}{rgb}{0,0,0.4}
\usepackage{hyperref}
\hypersetup{
    breaklinks=true,
    colorlinks=true,
    linkcolor=linkcolor,
    urlcolor=linkcolor,
    citecolor=black,
    filecolor=black,
    %filecolor=blue,
    pdfmenubar=true,
    pdftoolbar=true,
    bookmarksdepth=3   % Uncomment (and tweak) for PDF bookmarks with more levels than the TOC
    }

\setcounter{tocdepth}{2}  % levels in table of contents

\setcounter{topnumber}{2}
\setcounter{bottomnumber}{2}
\setcounter{totalnumber}{4}
\renewcommand{\topfraction}{0.95}
\renewcommand{\bottomfraction}{0.95}
\renewcommand{\textfraction}{0}
\renewcommand{\floatpagefraction}{0.75}
\clubpenalty = 10000
\widowpenalty = 10000

\raggedbottom
\makeindex
\usepackage[totoc]{idxlayout}   % for index in the toc
\usepackage[nottoc]{tocbibind}  % for references/bibliography in the toc

\begin{document}

\thispagestyle{empty}

\begin{center}
{\LARGE\bf
\begin{spacing}{1.25}
Computing Across the Disciplines: Study Programs in Computational Science and Data Science
\end{spacing}
}
\end{center}

\begin{center}
{\bf A comprehensive path from undergraduate studies to graduate studies (Master and PhD levels) in Computational Science and Data Science at the University of Oslo }\\ [0mm]
\end{center}


\vspace{1cm}


\section*{Executive Summary}


\section*{Introduction and Background}




\section*{Educational programs in Computational Science and Data Science}


\begin{enumerate}
\item An undergraduate program in Computational Science and Data Science.

\item We have already (from fall 2018) two new Master of Science programs in Computational Science and Data Science dailored to STEM fields. Should we merge these two programs into one and have one coordinating unit? 

\item Develop a PhD program in Computational Science and Data Science.

\subsection*{Bachelor of Science in Computational Science and Data Science}

\subsection*{Master of Science in Computational Science and Data Science}

\subsection*{PhD in Computational Science and Data Science}


\end{enumerate}










\section*{Appendix:  Outline of degree programs and courses}

\subsection*{Themes for a CS+DS Bachelor of Science program}

Here we have a list of themes which could enter the common part of the CS+DS bachelor program.  
Semesters are not indicated. Courses could also be modularized in 5 ECTS units.



\paragraph{Mathematics}
Here we can think of the topics and themes covered by the central  Math courses we have presently.
30 ECTS in total
\begin{enumerate}
\color{red}
    \item MAT1100
    \item MAT1110
    \item MAT1120
\end{enumerate}

\paragraph{Programming}
Here we can think of the topics and themes covered by the existing central programing courses.
20 ECTS plus 10 elective ECTS (also smaller 5 ECTS modules)
\begin{enumerate}
    \item \textcolor{red}{IN1900}
    \item \textcolor{red}{IN1910}
    \item IN3200, HPC, elective
\end{enumerate}


\paragraph{Computational Science}
20 ECTS + 10 ECTS elective (also smaller 5 ECTS modules)
\begin{enumerate}
    \item \textcolor{red}{Numerical Methods I}
    \item \textcolor{red}{Numerical Methods II}
    \itemNumerical Methods III
\end{enumerate}

\paragraph{Data Science}
20 ECTS + 10 ECTS elective (also smaller 5 ECTS modules)
\begin{enumerate}
    \item \textcolor{red}{Data Analysis, AI and Statistics I}
    \item \textcolor{red}{Data Analysis, AI and Statistics II}
    \item Data Analysis, AI and Statistics III
\end{enumerate}




\paragraph{Degree programs.}

The MNFak  offers from fall 2018 two new programs at the Master of Science level in Computational Science and Data Science. These programs are
\begin{enumerate}
\item \href{{http://www.uio.no/english/studies/programmes/computational-science-master/index.html}}{Computational Science}, start fall 2018

\item \href{{http://www.uio.no/english/studies/programmes/datascience-master/index.html}}{Data Science}, start fall 2018

\item We may consider merging these two programs and have one coordinating unit.
\item Develop an all university PhD program in Computational Science and Data Science by fall 202X?


\

\item Develop a bachelor program in Computational Science and Data Science by 2023.
\end{enumerate}

\noindent

\paragraph{Courses.}
There are several existing and planned courses which could be offered by the new department.
These are:
\begin{enumerate}

\item New courses on advanced data analysis and machine learning including for example:

\noindent
Many of these courses, if properly modularized, can be offered as intensive training courses and programs. In particular, such courses will be attractive for both the private and public sectors. The following courses could be offered
\begin{enumerate}
\item Introductory Scientific Python

\item Advanced Scientific Python

\item Data Science and visualization

\item Applied numerical mathematics

\item Computational finance

\item Big data graph analysis

\item Supervised machine learning with scikit-learn and TensorFlow

\item Unsupervised machine learning with scikit-learn

\item Data-driven entrepreneurship

\item Courses tailored to the needs of specific companies

\item and more specialized modules
\end{enumerate}


\end{document}


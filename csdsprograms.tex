\documentclass[oneside,final,10pt]{article}

\listfiles               %  print all files needed to compile this document

\usepackage{relsize,makeidx,color,setspace,amsmath,amsfonts,amssymb}
\usepackage[table]{xcolor}
\usepackage{bm,ltablex,microtype}
\usepackage[pdftex]{graphicx}

\usepackage[T1]{fontenc}
%\usepackage[latin1]{inputenc}
\usepackage{ucs}
\usepackage[utf8x]{inputenc}

\usepackage{lmodern}         % Latin Modern fonts derived from Computer Modern

% Hyperlinks in PDF:
\definecolor{linkcolor}{rgb}{0,0,0.4}
\usepackage{hyperref}
\hypersetup{
    breaklinks=true,
    colorlinks=true,
    linkcolor=linkcolor,
    urlcolor=linkcolor,
    citecolor=black,
    filecolor=black,
    %filecolor=blue,
    pdfmenubar=true,
    pdftoolbar=true,
    bookmarksdepth=3   % Uncomment (and tweak) for PDF bookmarks with more levels than the TOC
    }

\setcounter{tocdepth}{2}  % levels in table of contents

\setcounter{topnumber}{2}
\setcounter{bottomnumber}{2}
\setcounter{totalnumber}{4}
\renewcommand{\topfraction}{0.95}
\renewcommand{\bottomfraction}{0.95}
\renewcommand{\textfraction}{0}
\renewcommand{\floatpagefraction}{0.75}
\clubpenalty = 10000
\widowpenalty = 10000

\raggedbottom
\makeindex
\usepackage[totoc]{idxlayout}   % for index in the toc
\usepackage[nottoc]{tocbibind}  % for references/bibliography in the toc

\begin{document}

\thispagestyle{empty}

\begin{center}
{\LARGE\bf
\begin{spacing}{1.25}
Computing Across the Disciplines: Study Programs in Computational Science and Data Science at the University of Oslo
\end{spacing}
}
\end{center}

\begin{center}
{\bf A comprehensive path from undergraduate studies to graduate studies (Master and PhD levels) in Computational Science and Data Science at the University of Oslo }\\ [0mm]
\end{center}


\vspace{1cm}


\section*{Executive Summary}

We propose to establish, with startup Fall semester 2023, a new bachelor/undergraduate program in Computational Science and Data Science at the University of Oslo. 
The new undergraduate program is meant to be a multidisciplinary program with  a strong methodological base of courses in Mathematics, Statistics, Computational Science and Data Science, with a suggested number of ECTS of 90-100. \textcolor{red} {Merk: I f\o{}lge v\aa{}r studieadministrasjon skal et Bachelor-program ha 120 ECTS obligatoriske studiepoeng. Men det trenger ikke \aa{} v\ae{}re fellesemner, men en liste av typen: To av f\o{}lgende emner:} The remaining credits will allow students to specialize in different disciplines, like Bioscience, Chemistry, Geoscience, Informatics, Physics and other physical sciences. The program's multidisciplinary nature entails thus a close collaboration between departments at the Faculty of Mathematics and Natural Science.  

We propose also that the two existing Master of Science programs, Computational Science and Data Science, merge into one Master of Science program with the same name as the undergraduate program. The new Master of Science program will include all the study directions from the previous two Master of Science programs. Planned startup for the new program is also Fall 2023. 

Combined with the newly started {\bf CompSci} PhD program, our proposal is meant to enhance the visibility of Computational Science and Data Science at the University of Oslo, from undergraduate to graduate studies. It will also align well with many of the new research and educational initiatives on Computational Science and Data Science at the university of Oslo, in particular the newly established {\bf dScience} center. 



\section*{Introduction and Background}

Computational Science and Data Science play a central role in scientific investigations and are central to innovation in most domains of our lives. These fields underpin the majority of today's technological, economic and societal feats. We have entered an era in which huge amounts of data offer enormous opportunities, but only to those who are able to harness them. The 3rd Industrial Revolution will alter significantly the demands on the workforce. To adapt a highly-qualified workforce to upcoming challenges requires strong fundamental bases in STEM fields. Computational Science and Data Science can provide such bases at all stages. Most of our students at both the undergraduate and the graduate level are unprepared to use computational modeling, data science, and high performance computing – skills valued by a very broad range of employers. 
The above developments, needs and future challenges, as well as the exciting developments which are now taking place within quantum computing, quantum information and data driven discoveries (data analysis and machine learning) will play an essential role in shaping future technological developments. Most of these developments require true cross-disciplinary approaches.  In order to meet many of these challenges we propose here that the Faculty of Mathematics and Natural Science at the University of Oslo, establishes, with start Fall 2023, a new undergraduate program in Computational Science and Data Science.

Furthermore, we propose also that the two Master of Science programs, Computational Science and Data Science merge into one programs. This will enhance the visibility of the two fields and most likely attract many new students. Combined with the newly established {\bf CompSci} PhD program, this has the potential to offer new educational research paths in computational science and data science. 


\section*{Brief overview of Computational and Data Sciences}

Data science focuses on the development of tools designed to find trends within data sets that help scientists who are challenged with massive amounts of data to assess key relations within those data sets. These key relations provide hooks that allow scientists to identify models which, in turn, facilitate making accurate predictions in complex systems. For example, a key data science goal on the biological side would be better care for patients (e.g., personalized medicine). Given a patient’s genetic makeup, the proper data-driven model would identify the most effective treatment for that patient. 

Scientific computing focuses on the development of predictive computer models of the world around us. As study of physical phenomena through experimentation has become impossible, impractical and/or expensive, computational modeling has become the primary tool for understanding—equal in stature to analysis and experiment. Although we can now design an entire commercial aircraft through simulation alone (e.g., the Boeing 777), there are many fundamental problems in science and engineering that are beyond the scope of modern computers with current computational methods. The discipline of scientific computing is the development of new methods that make challenging problems tractable on modern computing platforms, providing scientists and engineers with key windows into the world around us.

The University of Oslo has a multitude of exciting research directions in Computational and Data Sciences, spanning from basic research in Artificial Intelligence and Machine Learning to applications in a great variety of disciplines. Many important recent advances in our understanding of the physical world have been driven by large-scale computational modeling and data analysis, for example, the 2012 discovery of the Higgs boson, the 2013 Nobel Prize in chemistry for computational modeling of molecules, and the 2016 discovery of gravitational waves.

The new Bachelor of Science program will give students a formal methodological basis in Mathematics, Statistics, Computational Science and Data Science and prepare them for further studies in these fields as well as other disciplines where such competences and knowledge play a central role. 



\section*{Educational programs in Computational Science and Data Science}


\subsection*{Bachelor of Science in Computational Science and Data Science}

The new undergraduate program in Computational Science and Data Science will offer a basic training in central elements in Mathematics, Computational Science, Data Science, Statistics and Informatics. 
The core part of the program is made up of central courses (see description in the appendix) in the above fields. Most of these courses are already available, while some courses will have to be developed in close collaboration with the involved departments at the Faculty for Mathematics and Natural Science.
The suggested core block of courses is set to 90-100 ECTS and will be compulsory for all students entering the program. These courses will also be the main requirements for admission to
 all study directions in the new Computational Science and Data Science Master of Science program.


The remaining 80 ECTS are elective and students can select from a large range of disciplines such as Mathematics, Statistics, Physics, Chemistry, Geoscience, Bioscience, Astrophysics and Informatics. 
% Discussion point:  should we also emphasize topics from the 
%Recommended blocks of courses, qualifying for additional master programmes will be provided.
With the core block of courses combined with recommended combinations of additional courses,  students should be able to apply for most of the Master of Science programs at the Faculty for Mathematics and Natural Science. 
%This program will also allow students to be qualified for all study directions in the new Computational Science and Data Science Master of Science program.

The suggested compulsory courses are listed in the appendix.

\subsection*{Master of Science in Computational Science and Data Science}

Presently the Faculty of Mathematics and Natural Science at the University of Oslo offers two Master of Science programs. One in Computational Science (CS) with ten study directions and one in Data Science (DS) with five specialisations. The CS program includes almost all disciplines at the Faculty of Mathematics and Natural Science. Seven departments are involved through ten study directions in Astrophysics, Bioinformatics, Bioscience, Chemistry, Geoscience, Imaging and Biomedical Computing, Materials Science, Mathematics and Mechanics and Physics. These study directions cover both applications and methodological developments. The DS Master of Science program is a collaboration between the Departments of Informatics and Mathematics and has its focus on developments of methods pertinent to Data Science, Mathematics  and Statistics.  
The program offers five specializations, 
Statistics and Machine Learning, Database Integration and Semantic Web, 
Data Science and Life Science, Language Technology and Digital Image Processing. 


In order to enhance the visibility of Computational Science and Data Science we propose to merge these two programs into one program, maintaining the study directions and specialisations. Within the new program we propose two methodological directions towards computational science and data science, respectively in addition to a more applied direction, combining both computational science and data science tools towards other fields. Compared to the existing master program in Computational Science, the new program will have a stronger focus on methodology while at the same time keeping the existing 10 study directions within the applied direction. Compared to the existing master program in Data Science, the methodological focus is kept but with the additional possibility of more applied directions.

The Bachelor and Master of Science programs
will most likely increase the visibility of Computational and Data Science at the University of Oslo, and hopefully lead to a larger recruitment of interested students. 

%\subsection*{PhD in Computational Science and Data Science}
%This needs to be discussed.









\section*{Appendix:  Outline of degree programs and courses}

\subsection*{Suggested Themes and Courses for a CS+DS Bachelor of Science program}

Here we have a list of courses and themes which could enter the common part of the CS+DS bachelor program.  
Semesters are not indicated. Courses in red are suggested as compulsory.
Many of the courses exist already while other courses need to be developed. 


\paragraph{Mathematics}
We suggest that the central topics and themes covered by the existing central  Math courses listed here, are incorporated as compulsory courses. They correspond to 
30 ECTS in total and are
\begin{enumerate}
\color{red}
    \item \textcolor{red}{MAT1100}, 1st year
    \item \textcolor{red}{MAT1110}, 1st year
    \item \textcolor{red}{MAT1120}, 2nd year
\end{enumerate}

\paragraph{Programming and computer science}
Programming is central to any form of computational and data science and is included with 20 obligatory ECTS in the proposed program. The form of programming needed by CS/DS students is less obvious. At least today, many CS/DS practioners/researchers mostly write small scripts, typically consisting of a straightforward sequence of numeric computations. A central question for the program is whether and to what degree (all or a portion of) CS/DS students should gain competence on how to relate constructively to the use of data/computational science in larger/more complex software systems and whether they should have competence on how to analyze and improve computational efficiency. 

A question we have left open for now is whether the course variants IN1900/IN1910 should be compulsory for all students. This would ensure an identical programming foundation for all students that later courses could build on, or whether students with stronger interests in the direction of informatics (or integration of CS/DS in larger systems) should be allowed to alternatively choose the course combination IN1000/IN1010. Specifically, a challenge with allowing the IN1000/IN1010 combination would be that such students would gain very limited practical experience with libraries for numerical analysis, which might not work well together with the remaining course combination.

A second question is whether CS/DS students should be expected to have one obligatory/recommended course in an informatics direction, beyond the two programming courses. An introductory course in algorithms could be such a candidate. 

Students have an interest in strengthening their competence in the informatics direction would have a range of courses at IFI that could be elected, including: IN2010, IN2040, IN2080, IN2090, IN2100, IN3020, IN3030, IN3040, IN3110, IN3130 and IN3240.


\begin{enumerate}
    \item \textcolor{red}{IN1900} 1st year, 10 ECTS (alternatively allow IN1000 to be chosen).
    \item \textcolor{red}{IN1910} (alternatively allow IN1010 to be chosen), 10 ECTS
    \item IN2010, IN2040, IN2080, IN2090, IN2100, IN3020, IN3030, IN3040, IN3110, IN3130 and IN3240, elective. 10 ECTS for each course. 
\end{enumerate}


\paragraph{Computational Science/Numerical Methods}
We propose that the Computational Science axis of the program is represented by courses in Numerical Mathematics plus elective courses in for example Computational Physics and similar other more senior undergraduate courses. The Computational Science axis has 
20 ECTS which are compulsory plus  10 ECTS which are  elective. The suggested courses are
\begin{enumerate}
    \item \textcolor{red}{Numerical Methods I} Could be revised version of MAT-INF1100, first year \textcolor{red}{MI og FI er i diskusjon om det nye f\o{}rstesemesteremnet i fysikk, som er ment \aa{} erstatte MAT-INF1100. Det kan kanskje v\ae{}re en ide \aa{} se dette i sammenheng.}
    \item \textcolor{red}{Numerical Methods II}, second year, could be MAT3110
    \item Numerical Methods III, elective,  third year, FYS3150, Computational Physics, third year and other alternatives. 
\end{enumerate}

\paragraph{Probability and Statistics}
20 ECTS + 10 ECTS elective (also smaller 5 ECTS modules). Here we can think of courses tailored to specific disciplines as well. 

 In order to understand methods within data science (or perhaps here restricted to data analysis), uncertainty, random variables and probability are important concepts that needs to be learning. For most students these are new concepts that they use time to learn properly. 
\begin{enumerate}
    \item \textcolor{red}{Introduction to probability and statistics}, first year, preferable a revised version of STK1100.
    
    An introductory course (10 ECTS) should introduce probability theory (including conditional probabilities and Bayes theorem, some common distributions, expectations, variances, covariances), statisitcal inference (the likelihood function, parametric modelling, ML estimation, bias in estimation, confidence intervals Bootrapping, some simple models). In order to make the course attractive for several bachelor programs, the course should include good examples/case studies from substance disciplines.
    
    \textcolor{red}{Den samme merknaden her som under MAT-INF1100. MI sitt standpunkt er at FYS 113x (eller FYS-STK 113x) er mer eller mindre et rent statistikkemne og b\o{}r undervises av MI. Kan dette sees i sammenheng?}
    \item \textcolor{red}{Statistical learning and machine learning}, 2nd year, preferable STK2100
    
    A second course (10 ECTS) should focus on statistical/machine learning methods, starting with linear models (going a bit in depth on this) and then demonstrating \emph{some} advanced methods (e.g nearest neighbor, neural network). Possibly also examples of unsupervised learning/data mining.

    \item Data Analysis, AI and Statistics III, third year
    
    Possible courses here are STK3100 - Generalized linear models,  
IN3050 – Introduction to Artificial Intelligence and Machine Learning, FYS-STK3155 – Applied Data Analysis and Machine Learning 
\end{enumerate}


The Department of Mathematics
(section for statistics and Data Science) is currently looking at modifications of STK1100 and STK2100 following the ideas above. A main change from the current set of courses is that STK2100 can be taken with only STK1100 as background (today it also is built on STK1110).

\paragraph{Discipline oriented courses}

These can be specific courses that lead to the requirements of a bachelor degree in life science, physics, chemistry etc. Most bachelor programs have a requirement of 60-90 ECTS within the specific fields. As an example, the Physics and Astronomy bachelor requires 60 ECTS in mathematics and programming and 70 ECTS in physics courses, plus 10 ECTS in advanced topics and 40 elective ECTS.  
Chemistry and Biochemistry requires 100 ECTS. 





\end{document}

